\documentclass[11pt,a4paper]{article}
\usepackage[utf8]{inputenc}
\usepackage{amsmath}
\usepackage{amsfonts}
\usepackage{amssymb}
\usepackage{graphicx}
\usepackage[ngerman]{babel}
\author{Lennart Karsten}
\title{Portierung einer JavaScript Fullstack Anwendung, in ein Microservice Frontend\\
	\vspace{3mm}\large Master Grundprojekt
}
\date{}


\begin{document}
	\maketitle
	\tableofcontents
	\newpage
	
	
	\section{Einleitung}
	
	\section{(Grundlagen)}
	
	\section{Analyse}
	\subsection{(Stakeholders)}
	\subsection{Use-cases}
	\subsection{Gap-Analyse}
	
	\section{Software Design}
	\subsection{Anforderungen}
	\subsubsection{Funktionale Anforderungen}
	\subsubsection{Nicht-funktionale Anforderungen}
	\subsection{Workflow}
	\subsection{Schnittstellen}
	

	\section{Umsetzung}
	\subsection{Technologien}
	\subsection{Portierung der Komponenten}
	\subsubsection{Import}
	\subsubsection{Data View}
	\subsubsection{Mapping}
	\subsection{Fehlerbehandlung}
	
	\section{Schluss}
	\subsection{Zusammenfassung}
	\subsection{Ausblick}
	
	\newpage
	\textbf{Grundlagen}\\
	Wie besprochen werde ich diesen Teil, falls notwendig mit Inhalt füllen und anderenfalls weglassen.\\
	
	\textbf{Analyse}\\
	In der Analyse möchte ich anhand der Stakeholder die Use-Cases ermitteln und diese in einer Gap-Analyse mit der alten Implementierung vergleichen.\\
	Ich bin mir nicht sicher, ob ich die Stakeholder als separaten Bereich aufführen soll, oder ob mir dafür der Inhalt fehlt.\\
	
	\textbf{Software Design}\\
	Im Design Abschnitt werde ich anhand der Analyse Anforderungen aufstellen und näher beschreiben.\\
	Der Workflow soll den Weg des Benutzers durch das System beschreiben und auf die Abhängigkeiten der 4 Phasen eingehen. Mit dem Namen der Überschrift bin ich noch nicht sicher.\\
	Die Schnittstellen, welche die WebUI konsumiert sollen beschrieben werden. Gerade in Bezug auf Restfullness der Endpunkte sehe ich Überschneidungen mit Mitjas Arbeit.\\
	
	\textbf{Umsetzung}\\
	Dieser Abschnitt beschreibt die eingesetzten Technologien und stellt den Bezug zu MS her. Der Technologie Teil soll dies erfüllen, ggf. gestützt durch einen Grundlagen Teil.\\
	Der 2. Teil der Umsetzung beschreibt, wie die einzelnen Komponenten portiert wurden.\\
	Fehlerbehandlung umfasst den Umgang mit Fehlern bei API Aufrufen und nicht vorhandener Backend Services.\\

	\textbf{Schluss}\\
	Im Ausblick werde ich auf das fehlende Gruppen und Sicherheitskonzept eingehen und somit mein Hauptprojekt einleiten.
	
	
\end{document}